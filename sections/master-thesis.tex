\section{Master thesis}
\cvitem{Title}{\emph{Automatic hyphenation---a rule-based approach to hyphenation}}
\cvitem{Description}{
 % TODO: Description from abstract. Needs to be re-written
 Current systems for hyphenation is based on pattern recognicion, include systems used for norwegian. And most systems, including most proprietary systems, are based on the algorithm developed by Franklin Mark Liang. For enligsh/american, where hyphenation is based on pronounciation, this works well. Hyphenation in norwegian is a rule based system, with two main rules, and about six sub-rules, describing where it's allowed to hyphenate. The norwegian language also says something about where it is best to hyphenate, which previous systems can't take into account. The hypothesis for this thesis was that it would be possible to achieve better hyphenation results, by implementing a word/language/lexical parser and with this information applying the hyphenation rules. The thesis tries to answer this problem, and also implements a rule based hyphenator in ruby. \href{https://www.duo.uio.no/handle/10852/44768?show=full}{Click here to read.}
 Supervisor was Dag Langmyhr.
}
